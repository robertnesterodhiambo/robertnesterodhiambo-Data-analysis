% Options for packages loaded elsewhere
\PassOptionsToPackage{unicode}{hyperref}
\PassOptionsToPackage{hyphens}{url}
%
\documentclass[
  12pt,
]{article}
\usepackage{amsmath,amssymb}
\usepackage{iftex}
\ifPDFTeX
  \usepackage[T1]{fontenc}
  \usepackage[utf8]{inputenc}
  \usepackage{textcomp} % provide euro and other symbols
\else % if luatex or xetex
  \usepackage{unicode-math} % this also loads fontspec
  \defaultfontfeatures{Scale=MatchLowercase}
  \defaultfontfeatures[\rmfamily]{Ligatures=TeX,Scale=1}
\fi
\usepackage{lmodern}
\ifPDFTeX\else
  % xetex/luatex font selection
\fi
% Use upquote if available, for straight quotes in verbatim environments
\IfFileExists{upquote.sty}{\usepackage{upquote}}{}
\IfFileExists{microtype.sty}{% use microtype if available
  \usepackage[]{microtype}
  \UseMicrotypeSet[protrusion]{basicmath} % disable protrusion for tt fonts
}{}
\makeatletter
\@ifundefined{KOMAClassName}{% if non-KOMA class
  \IfFileExists{parskip.sty}{%
    \usepackage{parskip}
  }{% else
    \setlength{\parindent}{0pt}
    \setlength{\parskip}{6pt plus 2pt minus 1pt}}
}{% if KOMA class
  \KOMAoptions{parskip=half}}
\makeatother
\usepackage{xcolor}
\usepackage[margin=1in]{geometry}
\usepackage{graphicx}
\makeatletter
\def\maxwidth{\ifdim\Gin@nat@width>\linewidth\linewidth\else\Gin@nat@width\fi}
\def\maxheight{\ifdim\Gin@nat@height>\textheight\textheight\else\Gin@nat@height\fi}
\makeatother
% Scale images if necessary, so that they will not overflow the page
% margins by default, and it is still possible to overwrite the defaults
% using explicit options in \includegraphics[width, height, ...]{}
\setkeys{Gin}{width=\maxwidth,height=\maxheight,keepaspectratio}
% Set default figure placement to htbp
\makeatletter
\def\fps@figure{htbp}
\makeatother
\setlength{\emergencystretch}{3em} % prevent overfull lines
\providecommand{\tightlist}{%
  \setlength{\itemsep}{0pt}\setlength{\parskip}{0pt}}
\setcounter{secnumdepth}{-\maxdimen} % remove section numbering
\ifLuaTeX
  \usepackage{selnolig}  % disable illegal ligatures
\fi
\usepackage{bookmark}
\IfFileExists{xurl.sty}{\usepackage{xurl}}{} % add URL line breaks if available
\urlstyle{same}
\hypersetup{
  pdftitle={Voting System For General Elections},
  pdfauthor={Stephanie Amondi and Mitchelle Kerubo; Mitchelle Kerubo},
  hidelinks,
  pdfcreator={LaTeX via pandoc}}

\title{Voting System For General Elections}
\author{Stephanie Amondi and Mitchelle Kerubo \and Mitchelle Kerubo}
\date{July 04, 2024}

\begin{document}
\maketitle

{
\setcounter{tocdepth}{2}
\tableofcontents
}
\usepackage{times}
\renewcommand{\familydefault}{\sfdefault}
\begin{center}
  \huge\textbf{Young Scientist Kenya} \\
  \vspace{0.5cm}
  \Large\textbf{Category: Technology} \\
  \vspace{0.5cm}
  \large\textbf{School: Asumbi Girls High School} \\
\end{center}

\section{Declaration}\label{declaration}

We, Stephanie Amondi and Mitchelle Kerubo, hereby declare that this
project, titled ``Developing the Election Process,'' is our original
work. We have conducted all research, design, and development
independently, without unauthorized assistance. All sources and
references used in the development of this project have been properly
cited and acknowledged. This project is a result of our own efforts and
has not been submitted for assessment in any other context or
institution.

We affirm that we have adhered to all guidelines and ethical standards
set by our school and any other relevant authorities. We take full
responsibility for the authenticity and integrity of our work and are
committed to upholding the highest standards of academic honesty and
integrity. We appreciate the guidance and support provided by our
teachers and mentors, but we confirm that the work presented here is a
product of our own initiative and effort.

\newpage

\section{Abstract}\label{abstract}

In Kenya, electoral processes have been marred by numerous instances of
corruption, undermining public trust and the integrity of the voting
system. This project, ``Developing the Election Process,'' aims to
address these issues by proposing a robust, transparent, and efficient
voting system. Our research method involved a comprehensive analysis of
existing voting systems, identifying their vulnerabilities, and
designing a new framework that leverages blockchain technology to ensure
security and transparency. Through simulations and trials, we tested the
proposed system's resilience against common electoral malpractices. The
results demonstrated a significant reduction in opportunities for vote
tampering and increased transparency in the vote tallying process. This
project concludes that the adoption of blockchain technology in Kenya's
voting system can enhance electoral integrity, restore public trust, and
ensure fairer elections. Key methods include system analysis, blockchain
integration, and security testing. Our findings indicate that modern
technological solutions can effectively mitigate electoral corruption.

\end{document}
