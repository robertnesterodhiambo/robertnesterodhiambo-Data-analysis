% Options for packages loaded elsewhere
\PassOptionsToPackage{unicode}{hyperref}
\PassOptionsToPackage{hyphens}{url}
%
\documentclass[
]{article}
\usepackage{amsmath,amssymb}
\usepackage{iftex}
\ifPDFTeX
  \usepackage[T1]{fontenc}
  \usepackage[utf8]{inputenc}
  \usepackage{textcomp} % provide euro and other symbols
\else % if luatex or xetex
  \usepackage{unicode-math} % this also loads fontspec
  \defaultfontfeatures{Scale=MatchLowercase}
  \defaultfontfeatures[\rmfamily]{Ligatures=TeX,Scale=1}
\fi
\usepackage{lmodern}
\ifPDFTeX\else
  % xetex/luatex font selection
\fi
% Use upquote if available, for straight quotes in verbatim environments
\IfFileExists{upquote.sty}{\usepackage{upquote}}{}
\IfFileExists{microtype.sty}{% use microtype if available
  \usepackage[]{microtype}
  \UseMicrotypeSet[protrusion]{basicmath} % disable protrusion for tt fonts
}{}
\makeatletter
\@ifundefined{KOMAClassName}{% if non-KOMA class
  \IfFileExists{parskip.sty}{%
    \usepackage{parskip}
  }{% else
    \setlength{\parindent}{0pt}
    \setlength{\parskip}{6pt plus 2pt minus 1pt}}
}{% if KOMA class
  \KOMAoptions{parskip=half}}
\makeatother
\usepackage{xcolor}
\usepackage[margin=1in]{geometry}
\usepackage{graphicx}
\makeatletter
\def\maxwidth{\ifdim\Gin@nat@width>\linewidth\linewidth\else\Gin@nat@width\fi}
\def\maxheight{\ifdim\Gin@nat@height>\textheight\textheight\else\Gin@nat@height\fi}
\makeatother
% Scale images if necessary, so that they will not overflow the page
% margins by default, and it is still possible to overwrite the defaults
% using explicit options in \includegraphics[width, height, ...]{}
\setkeys{Gin}{width=\maxwidth,height=\maxheight,keepaspectratio}
% Set default figure placement to htbp
\makeatletter
\def\fps@figure{htbp}
\makeatother
\setlength{\emergencystretch}{3em} % prevent overfull lines
\providecommand{\tightlist}{%
  \setlength{\itemsep}{0pt}\setlength{\parskip}{0pt}}
\setcounter{secnumdepth}{-\maxdimen} % remove section numbering
\ifLuaTeX
  \usepackage{selnolig}  % disable illegal ligatures
\fi
\IfFileExists{bookmark.sty}{\usepackage{bookmark}}{\usepackage{hyperref}}
\IfFileExists{xurl.sty}{\usepackage{xurl}}{} % add URL line breaks if available
\urlstyle{same}
\hypersetup{
  hidelinks,
  pdfcreator={LaTeX via pandoc}}

\author{}
\date{\vspace{-2.5em}}

\begin{document}

\hypertarget{i-mean-and-standard-deviation}{%
\section{i) mean and standard
deviation}\label{i-mean-and-standard-deviation}}

Sample: 70, 45, 60, 50, 65, 40, 90, 55, 40, 50, 85, 40, 50, 65, 45, 60,
55, 45, 120, 70

\emph{Step 1: Calculate the Mean (Average)} To find the mean, sum up all
the numbers in the sample and then divide by the total number of values.

Sum of the numbers = 70 + 45 + 60 + 50 + 65 + 40 + 90 + 55 + 40 + 50 +
85 + 40 + 50 + 65 + 45 + 60 + 55 + 45 + 120 + 70 = 1120

Total number of values = 20

Mean = Sum of numbers / Total number of values Mean = 1120 / 20 Mean =
56

\emph{Step 2: Calculate the Standard Deviation} The formula for standard
deviation involves several steps:

\begin{enumerate}
\def\labelenumi{\alph{enumi}.}
\tightlist
\item
  Calculate the mean (which we already did).
\item
  Subtract the mean from each number in the sample to find the
  deviations.
\item
  Square each deviation.
\item
  Find the mean of the squared deviations.
\item
  Take the square root of the mean of the squared deviations.
\end{enumerate}

Let's calculate it step by step:

\begin{enumerate}
\def\labelenumi{\alph{enumi}.}
\item
  Mean = 56 (already calculated)
\item
  Deviations: (70 - 56), (45 - 56), (60 - 56), (50 - 56), (65 - 56), (40
  - 56), (90 - 56), (55 - 56), (40 - 56), (50 - 56), (85 - 56), (40 -
  56), (50 - 56), (65 - 56), (45 - 56), (60 - 56), (55 - 56), (45 - 56),
  (120 - 56), (70 - 56) Deviation = 14, -11, 4, -6, 9, -16, 34, -1, -16,
  -6, 29, -16, -6, 9, -11, 4, -1, -11, 64, 14
\item
  Square each deviation: Deviation\^{}2 = 196, 121, 16, 36, 81, 256,
  1156, 1, 256, 36, 841, 256, 36, 81, 121, 16, 1, 121, 4096, 196
\item
  Find the mean of the squared deviations: Mean of squared deviations =
  (196 + 121 + 16 + \ldots{} + 4096) / 20 Mean of squared deviations =
  663.6
\item
  Take the square root of the mean of the squared deviations: Standard
  deviation = sqrt(663.6) Standard deviation ≈ 25.73
\end{enumerate}

So, the mean of the sample is 56 and the standard deviation is
approximately 25.73.

\end{document}
